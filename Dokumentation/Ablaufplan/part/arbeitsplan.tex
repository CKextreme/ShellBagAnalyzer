\section{Arbeitsplan und Meilensteine}

\begin{longtable}{|p{0.75\textwidth}|p{0.2\textwidth}|}
	
	\hline
	\cellcolor{gray!25}\textbf{Arbeitsschritt} & \cellcolor{gray!25}\textbf{Zieldatum} \\
	\hline
	Auseinandersetzen mit grundlegenden Informationen über ShellBags und Verfassen des Grundlagenkapitels & 12.04.2020 \\
	\hline
	Anforderungsplanung: Entwickeln eines Entwurfs für den Aufbau des Tools und der Darstellung der ShellBag-Informationen & 12.04.2020  \\
	\hline
	Recherche zu einer geeigneten Entwicklungsumgebung und erstes Vertrautmachen mit dieser Entwicklungsumgebung & 19.04.2020 \\
	\hline
	Programmierung Teil 1: Schaffen eines Zuganges zur Windows-Registry und erstes Auslesen von Einträgen & 26.04.2020 \\
	\hline
	Programmierung Teil 2: Implementierung der Algorithmen & 10.05.2020 \\
	\hline
	Programmierung Teil 3: grafische Visualisierung der ShellBag-Informationen im Tool & 17.05.2020 \\
	\hline
	Programmierung Teil 4: Vornahme von Optimierungen und Verbesserungen am Tool  & 24.05.2020 \\
	\hline
	Programmierung Teil 5: Programmieren einer Lösung, um die ShellBag-Informationen aus dem Tool zu exportieren & 07.06.2020 \\
	\hline
	Programmierung Teil 6: Programmieren einer Lösung, um die ShellBag-Informationen aus der Windows-Registry zu löschen & 14.06.2020 \\
	\hline
	Testphase: Validierung des implementierten Tools durch den Test in einer aufgesetzten virtuellen Maschine und gegebenenfalls Vornahme von Optimierungen im Programm & 28.06.2020 \\
	\hline
	Überarbeitung und Abschluss der schriftlichen Belegarbeit & 03.07.2020 \\
	\hline
\end{longtable}

Das Verfassen der schriftlichen Belegarbeit soll parallel zum Implementieren des Tools erfolgen. Am Ende wird noch eine Überarbeitung und Korrektur vorgenommen. \\
Die zuvor abgebildete Tabelle zeigt die einzelnen Arbeitsschritte im Rahmen des Projektes. Die zugewiesenen Datumsangaben sind Zielsetzungen, bis wann die einzelnen Schritte umgesetzt werden sollen. Es ist durchaus möglich, dass einige  Arbeitsschritte mehr Zeit in Anspruch nehmen werden, andere wiederum weniger Zeit. Dies ergibt sich erst im Laufe der Arbeit. Aus diesem Grund handelt es sich hierbei um einen vorläufigen Arbeitsplan.
\vspace{1em}