% This is the main tex file
% Please leave the following settings unchanged!
% ___________________________________________________________________
\documentclass{scrartcl}%


\usepackage{amsmath}%
\usepackage{amsfonts}%
\usepackage{amssymb}%
\usepackage{graphicx}
\usepackage{geometry}
%\usepackage{hyperref}
\usepackage{color}
\usepackage{caption}
\usepackage{ngerman} 
\usepackage[utf8]{inputenc}
\usepackage{tabularx}
\usepackage{longtable}
\usepackage{colortbl}
\usepackage{float}
\usepackage{xcolor}
\usepackage[onehalfspacing]{setspace}

%  Definition of some mathematical environments
\newtheorem{theorem}{Theorem}
\newtheorem{acknowledgement}[theorem]{Acknowledgement}
\newtheorem{algorithm}[theorem]{Algorithm}
\newtheorem{axiom}[theorem]{Axiom}
\newtheorem{case}[theorem]{Case}
\newtheorem{claim}[theorem]{Claim}
\newtheorem{conclusion}[theorem]{Conclusion}
\newtheorem{condition}[theorem]{Condition}
\newtheorem{conjecture}[theorem]{Conjecture}
\newtheorem{corollary}[theorem]{Corollary}
\newtheorem{criterion}[theorem]{Criterion}
\newtheorem{definition}[theorem]{Definition}
\newtheorem{example}[theorem]{Example}
\newtheorem{exercise}[theorem]{Exercise}
\newtheorem{lemma}[theorem]{Lemma}
\newtheorem{notation}[theorem]{Notation}
\newtheorem{problem}[theorem]{Problem}
\newtheorem{proposition}[theorem]{Proposition}
\newtheorem{remark}[theorem]{Remark}
\newtheorem{solution}[theorem]{Solution}
\newtheorem{summary}[theorem]{Summary}
\newenvironment{proof}[1][Proof]{\textbf{#1.} }{\ \rule{0.5em}{0.5em}}

\parindent 0.0mm
\geometry{a4paper,left=30mm,right=25mm, top=25mm, bottom=2cm}

% ___________________________________________________________________
% ___________________________________________________________________


\begin{document}
	\title{Ablaufplan zum Softwarepraktikum: \\ Implementierung eines Tools zur forensischen Analyse von ShellBags unter Windows 10}
	\author{Conny Karras und Anna-Lena Totzauer
		\\
		\\ \small{Cybercrime/Cybersecurity, CY19wC-M}
		\\ \small{Betreuer: Prof. Dr. rer. nat. Labudde}
		\\ \small{Bearbeitungszeitraum: April 2020 - Juli 2020}
		\\ \footnotesize{}}

	\date{}
	\maketitle
	

	% Thema
	\section{Thema und Zielsetzung}
Ziel dieser Arbeit soll es sein, ein Tool zu entwickeln, welches eine forensische Analyse von ShellBags unter Windows 10 ermöglicht. Basierend auf der Bachelorarbeit von Anna-Lena Totzauer zum Thema \glqq Forensische Analyse von Windows ShellBags\grqq{} sollen die in der Registry enthaltenen Informationen in einer für den Menschen unmittelbar lesbaren Form dargestellt werden. Darüber hinaus soll es möglich sein, die gefundenen Einträge zu exportieren sowie aus der Registry zu löschen.

	% Arbeistplan
	\section{Arbeitsplan und Meilensteine}

\begin{longtable}{|p{0.75\textwidth}|p{0.2\textwidth}|}
	
	\hline
	\cellcolor{gray!25}\textbf{Arbeitsschritt} & \cellcolor{gray!25}\textbf{Zieldatum} \\
	\hline
	Auseinandersetzen mit grundlegenden Informationen über ShellBags und Verfassen des Grundlagenkapitels & 12.04.2020 \\
	\hline
	Anforderungsplanung: Entwickeln eines Entwurfs für den Aufbau des Tools und der Darstellung der ShellBag-Informationen & 12.04.2020  \\
	\hline
	Recherche zu einer geeigneten Entwicklungsumgebung und erstes Vertrautmachen mit dieser Entwicklungsumgebung & 19.04.2020 \\
	\hline
	Programmierung Teil 1: Schaffen eines Zuganges zur Windows-Registry und erstes Auslesen von Einträgen & 26.04.2020 \\
	\hline
	Programmierung Teil 2: Implementierung der Algorithmen & 10.05.2020 \\
	\hline
	Programmierung Teil 3: grafische Visualisierung der ShellBag-Informationen im Tool & 17.05.2020 \\
	\hline
	Programmierung Teil 4: Vornahme von Optimierungen und Verbesserungen am Tool  & 24.05.2020 \\
	\hline
	Programmierung Teil 5: Programmieren einer Lösung, um die ShellBag-Informationen aus dem Tool zu exportieren & 07.06.2020 \\
	\hline
	Programmierung Teil 6: Programmieren einer Lösung, um die ShellBag-Informationen aus der Windows-Registry zu löschen & 14.06.2020 \\
	\hline
	Testphase: Validierung des implementierten Tools durch den Test in einer aufgesetzten virtuellen Maschine und gegebenenfalls Vornahme von Optimierungen im Programm & 28.06.2020 \\
	\hline
	Überarbeitung und Abschluss der schriftlichen Belegarbeit & 03.07.2020 \\
	\hline
\end{longtable}

Das Verfassen der schriftlichen Belegarbeit soll parallel zum Implementieren des Tools erfolgen. Am Ende wird noch eine Überarbeitung und Korrektur vorgenommen. \\
Die zuvor abgebildete Tabelle zeigt die einzelnen Arbeitsschritte im Rahmen des Projektes. Die zugewiesenen Datumsangaben sind Zielsetzungen, bis wann die einzelnen Schritte umgesetzt werden sollen. Es ist durchaus möglich, dass einige  Arbeitsschritte mehr Zeit in Anspruch nehmen werden, andere wiederum weniger Zeit. Dies ergibt sich erst im Laufe der Arbeit. Aus diesem Grund handelt es sich hierbei um einen vorläufigen Arbeitsplan.
\vspace{1em}
	
	% Umsetzung
	\section{Praktische Umsetzung}
Als Eingabe für die Software sollen die ShellBag-Schlüssel der Windows-Registry in ihrer ursprünglichen Form dienen. \\
Im Verarbeitungsschritt sollen diese Informationen gelesen und in eine lesbare Form umgewandelt sowie visualisiert werden. Darüber hinaus soll auch eine Möglichkeit implementiert werden, um die ShellBag-Informationen aus der Registry zu löschen. \\
Die Ausgabe soll eine grafische Visualisierung der ShellBag-Informationen zum Ziel haben. Dabei sollen die Informationen in einer für den Menschen unmittelbar lesbaren Form dargestellt werden, sodass direkt im Anschluss eine forensische Analyse stattfinden könnte. Weiterhin soll es möglich sein, diese Ausgabe zu exportieren.


\end{document}
