\section{Qualitätsanforderungen} \label{quali}
\vspace{0.5cm}
Nachfolgend werden die Qualitätsanforderungen an das Tool erläutert, welche bei der Implementierung berücksichtigt werden müssen, um eine optimale Eignung gewährleisten zu können.
\paragraph{Kompatibilität:}
Da die ShellBags in ihrer Form als Teil der Registry nur für Windows-Betriebssysteme existieren, kann die ShellBag-Live-Analyse auch nur auf Windows-Systemen erfolgen. Das implementierte Tool wird auch aufgrund der Nutzung des .NET Frameworks nur unter diesem Betriebssystem lauffähig sein \cite{netfw}.
\paragraph{Integrität:}
Im lesenden Modus der ShellBags aus der Windows-Registry dürfen Einträge nicht manipuliert, gelöscht oder eingefügt werden. Im Exportmodus sollte die hierarchische Struktur der ShellBags erkennbar sein und auch hier dürfen keine Manipulationen erfolgen.
\paragraph{Zuverlässigkeit:}
Aus forensischer Sicht soll das Tool auch bei fehlerhaften Einträgen eine zuverlässige Analyse und einen gewissen Teil-Output ermöglichen.
\paragraph{Benutzerfreundlichkeit:}
Das Programm sollte einfach gestaltet sein und auch Menschen mit geringen Kenntnissen über ShellBags einen leichten Einstieg ermöglichen.
\paragraph{Leistungseffizienz:}
Das Tool soll seine Ergebnisse nach einem angemessenen Zeitrahmen präsentieren können.
\paragraph{Portabilität:}
Es soll gewährleistet werden, dass die Software schnell zwischen mehreren Rechnern portiert werden kann, ohne dass dafür spezielle Installationen notwendig sind.






