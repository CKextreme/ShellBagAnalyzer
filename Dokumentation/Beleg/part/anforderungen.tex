\section{Qualitätsanforderungen}
Nachfolgend werden die Qualitätsanforderungen an das Tool erläutert, welche bei der Implementierung berücksichtigt werden müssen, um eine optimale Eignung gewährleisten zu können.
\paragraph{Kompabilität des Tools:}
Eine Live-Analyse der ShellBags ist nur auf einem Windows-Betriebssystem möglich, da sie als Teil der Registry nur in Windows-Systemen existieren. Eine Post-Mortem-Analyse der ShellBags mit dem implementierten Tool wird ebenfalls nur auf Windows-Betriebssystemen möglich sein aufgrund der Nutzung des .NET Frameworks, welches nur unter Windows lauffähig ist \cite{netfw}.
%lauffähigkeit vllt irrelevant, da nur onlineanalyse betrachtet werden
\paragraph{Sicherheit/Integrität:}
Im lesenden Modus der ShellBags aus der Windows-Registry dürfen Einträge nicht manipuliert, gelöscht oder eingefügt werden. Im Löschmodus darf keine fehlerhafte Löschung von Einträgen in der Registry erfolgen, da so die Funktionsfähigkeit des Systems eingeschränkt werden könnte.
\paragraph{Zuverlässigkeit:}
Aus forensischer Sicht soll das Tool auch bei einer größeren Registry oder fehlerhaften Einträgen eine zuverlässige Analyse und auch bei Fehlern einen gewissen Teil-Output ermöglichen.
\paragraph{Benutzerfreundlichkeit:}
Das Programm sollte einfach gestaltet sein und auch Menschen mit geringen Kenntnissen über ShellBags einen leichten Einstieg ermöglichen.
\paragraph{Leistungseffizienz:}
Das Tool soll seine Ergebnisse nach einem angemessenen Zeitrahmen präsentieren können.
\paragraph{Portabilität:}
Es soll gewährleistet werden, dass die Software schnell zwischen mehreren Rechnern portiert werden kann, ohne dass dafür spezielle Installationen notwendig sind.






