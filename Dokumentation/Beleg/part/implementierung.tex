\section{Praktische Umsetzung}
\vspace{0.5cm}
Im Anschluss erfolgt eine Beschreibung, wie bei der praktischen Umsetzung des Tools vorgegangen wurde. \\
\\
Grundlage für die Implementierung des Tools soll die integrierte Entwicklungsumgebung (IDE) Visual Studio 2019 von Microsoft darstellen \cite{vs}. Als Programmiersprache wird C\# mit dem .NET Framework verwendet. Bei dem .NET Framework, welches ein Teil der Microsoft Software-Plattform .NET ist, handelt es sich um eine umfangreiche Infrastruktur, in welcher Anwendungen programmiert, kompiliert, ausgeführt und verteilt werden können \cite[S.68]{bayer2008visual}. Dieses Framework ist ausschließlich auf Windows-Systemen verfügbar, das heißt, Anwendungen können nur unter Windows entwickelt und ausgeführt werden \cite{netfw}. \\
\\
Im Anschluss erfolgte die Implementierung der Live-Analyse der ShellBags. Im ersten Schritt wurde sich neben dem Aufbau der grafischen Benutzeroberfläche unter \textit{ShellBag.GUI} Zugang zu den ShellBag-Schlüsseln im Registrierungs-Editor unter \texttt{HKU$\backslash$SID\_User$\backslash$Software$\backslash$Classes$\backslash$ \newline Local Settings$\backslash$Software$\backslash$Microsoft$\backslash$Windows$\backslash$Shell$\backslash$BagMRU} bzw. unter \texttt{HKU$\backslash$SID\_User$\backslash$ \newline Software$\backslash$Microsoft$\backslash$Windows$\backslash$Shell} verschafft. Dies erfolgte in der Klasse \textit{ShellBag.Library. \newline ShellBagParser.cs}. Es wurde eine Auswahlmöglichkeit geschaffen, bei der man eine bestimmte SID wählen kann, dessen ShellBag-Schlüssel nun in das Tool eingelesen werden können. Darüber hinaus wurde die Beziehung zwischen der SID eines Benutzers und dem jeweiligen Benutzernamen identifiziert. Hintergrund für diesen Zusammenhang stellt zum einen der Befehl \glqq \texttt{whoami /user}\grqq{} dar, welcher die SID des aktuellen Benutzers ausgibt \cite{whoami}. Zum anderen gibt der Befehl \glqq \texttt{wmic useraccount where sid='<SID>' get name}\grqq{} den zur SID dazugehörigen Benutzernamen aus \cite{sid}. Diese Hilfsmethoden sind unter \textit{ShellBag.Library.Shell \newline Bags.ShellBagHelper.cs} festgeschrieben. Das Ergebnis des ersten Schrittes ist eine grafische Benutzeroberfläche, auf welcher die ShellBag-Schlüssel eines konkreten Benutzers aus dem Registrierungs-Editor eingelesen werden können. Die Darstellung der Informationen entspricht der Form, wie die Schlüssel im Registrierungs-Editor abgebildet werden, nur dass im implementierten Tool neben anderen ausschließlich die ShellBag-Schlüssel abgebildet und zusätzlich die Bezeichnungen der jeweiligen Ordner dargestellt sind, um sich bereits hier einen ersten Überblick verschaffen zu können. Darüber hinaus wird berechnet, wie viele Subkeys jedes Shell Item besitzt. Es besteht somit ein optischer Bezug zum Registrierungs-Editor, jedoch werden noch Zusatzinformationen abgebildet. Es erfolgte außerdem eine Implementierung der Ladezeit sowie der Anzahl aller Shell Items, um später die Leistungseffizienz bewerten zu können.

Nun wurden die Values der Shell Items ausgewertet. Für jedes Shell Item wurde dafür eine eigene Klasse unter \textit{ShellBag.Library.ShellBags.ShellItems} generiert, in der die spezifischen Informationen ausgelesen wurden. Diese Values wurden transparent im Hintergrund ausgewertet. Es erfolgte zunächst eine Zuordnung des Values zum passenden Subkey unter BagMRU. Die Zuordnung basiert darauf, dass der Value durch dieselbe Zahl beschrieben wird wie der passende Subkey unter BagMRU. Um zunächst zwischen den Arten von Shell Items zu unterscheiden, wurde der Class Type Indicator an Offset 0x02 betrachtet. Dies war notwendig, da die Shell Items verschiedene Strukturen aufweisen, je nachdem, um welche Art es sich handelt. Lautet der Class Type Indicator zum Beispiel 0x31, so handelt es sich um ein File Entry Shell Item und die Auswertung der Informationen erfolgt gemäß Tabelle \ref{fes}. \\
\\
Anschließend wurde die Exportfunktion im Tool implementiert. Dazu wurde eine eigene Klasse unter \textit{ShellBag.Library.ShellBags.Logging.FileExport.cs} generiert. Hier erfolgte eine separate Behandlung der ShellBags der NTUSER.dat und der USRCLASS.dat, das heißt, es wurde so programmiert, dass für beide Hives eine separate Textdatei angelegt wird. Die Darstellung der ShellBag-Informationen erfolgte in einer hierarchischen Form, sodass auch hier deutlich erkennbar ist, welche Elemente Kinder vom übergeordneten Knoten sind und welche Elemente sich auf gleicher Ebene befinden. Der Name der jeweiligen Textdatei wurde wie folgt gewählt: \glqq Export\_Hive-Name\_Datum\_Zeit.txt\grqq{}. Somit ist klar erkennbar, aus welchem Hive die exportierten ShellBags stammen und zu welcher Zeit der Export erfolgte.




%vllt noch ergänzen, welcher Schritt in welcher Methode/ Klasse bzw. in welcher Datei der Code zu finden ist --> zB auch Umwandlung GUID oder Zeitstempel: mit welcher Methode?


