\section{Praktische Umsetzung}
Im Anschluss erfolgt eine Beschreibung wie bei der praktischen Umsetzung des Tools vorgegangen wurde. \newline
Grundlage für die Implementierung des Tools soll die integrierte Entwicklungsumgebung (IDE) Visual Studio 2019 von Microsoft darstellen \cite{vs}. Als Programmiersprache wird C\# mit dem .NET Framework verwendet. Bei dem .NET Framework, welches ein Teil der Microsoft Software-Plattform .NET ist, handelt es sich um eine umfangreiche Infrastruktur, in welcher Anwendungen programmiert, kompiliert, ausgeführt und verteilt werden können \cite[S.68]{bayer2008visual}. Dieses Framework ist ausschließlich auf Windows-Systemen verfügbar, das heißt Anwendungen können nur unter Windows entwickelt und ausgeführt werden \cite{netfw}.

Im Anschluss erfolgte die Implementierung der Live-Analyse der ShellBags. Im ersten Schritt wurde sich neben dem Aufbau der grafischen Benutzeroberfläche Zugang zu den ShellBag-Schlüsseln im Registrierungs-Editor unter \texttt{HKU$\backslash$SID\_User$\backslash$Software$\backslash$Classes$\backslash$Local Set- \newline tings$\backslash$Software$\backslash$Microsoft$\backslash$Windows$\backslash$Shell$\backslash$BagMRU} bzw. unter \texttt{HKU$\backslash$SID\_User$\backslash$Software$\backslash$ \newline Microsoft$\backslash$Windows$\backslash$Shell} verschafft. Es wurde eine Auswahlmöglichkeit geschaffen, bei der man eine bestimmte SID wählen kann, dessen ShellBag-Schlüssel nun in das Tool eingelesen werden können. Darüber hinaus wurde die Beziehung zwischen der SID eines Benutzers und dem jeweiligen Benutzernamen identifiziert. Hintergrund für diesen Zusammenhang stellt zum einen der Befehl \glqq \texttt{whoami /user}\grqq{} dar, welcher die SID des aktuellen Benutzers ausgibt \cite{whoami}. Zum anderen gibt der Befehl \glqq \texttt{wmic useraccount where sid='<SID>' get name}\grqq{} den zur SID dazugehörigen Benutzernamen aus \cite{sid}. Das Ergebnis des ersten Schrittes ist eine grafische Benutzeroberfläche, auf welcher die ShellBag-Schlüssel eines konkreten Benutzers aus dem Registrierungs-Editor eingelesen werden können. Die Darstellung der Informationen entspricht der Form, wie die Schlüssel im Registrierungs-Editor abgebildet werden, nur dass im implementierten Tool neben anderen ausschließlich die ShellBag-Schlüssel abgebildet sind. Das Ziel in der Darstellung der Schlüssel bestand darin, einen Bezug zum Registrierungs-Editor herzustellen. Dies wurde durch die nahezu identische Darstellung der Schlüssel erreicht. Somit besteht ein optischer Bezug.


Nun wurden die Values der Shell Items ausgewertet. Diese Values wurden transparent im Hintergrund aus der Registry geladen und ausgewertet. Es erfolgte zunächst eine Zuordnung des Values zum passenden Subkey unter BagMRU, da die Values stets eine Ebene höher abgelegt sind als der Subkey selbst, das heißt der Value eines Subkeys unter BagMRU ist im Registrierungs-Editor im BagMRU-Schlüssel abgelegt. Die Zuordnung basiert darauf, dass der Value durch dieselbe Zahl beschrieben wird wie der passende Subkey unter BagMRU. Um zunächst zwischen den Arten von Shell Items zu unterscheiden, wurde der Class Type Indicator an Offset 0x02 betrachtet. Dies war notwendig, da die Shell Items verschiedene Strukturen aufweisen, je nachdem, um welche Art es sich handelt. Lautet der Class Type Indicator zum Beispiel 0x31, so handelt es sich um ein File Entry Shell Item und die Auswertung der Informationen erfolgt gemäß Tabelle \ref{fes}.



%welches Tool braucht man, welche voreinstellungen etc.
%Export-Funktion
%Lösch-Funktion




