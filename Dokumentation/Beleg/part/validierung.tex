\section{Validierung}
Um das implementierte Tool zu testen, wird eine virtuelle Maschine verwendet. Hierfür war es zunächst notwendig, die Virtualisierungssoftware VirtualBox \footnote{Download unter \url{https://www.chip.de/downloads/VirtualBox_23814448.html}, zuletzt verfügbar am 16.04.2020}  herunterzuladen. Im Anschluss war es möglich, bei Microsoft eine bereits fertige virtuelle Maschine (VM) mit Windows 10 als 90 Tage lang kostenlose Version \footnote{Download unter \url{https://developer.microsoft.com/en-us/microsoft-edge/tools/vms/}, zuletzt verfügbar am 16.04.2020}  herunterzuladen. Diese VM kann nach dem Entpacken problemlos in VirtualBox importiert und ausgeführt werden. \\
%welche Einstellungen noch vorgenommen, noch was installiert?, wie das Tool reingebkommen 
\\
Um zu testen, ob das Tool eine korrekte Auswertung der ShellBag-Informationen vornimmt, wird im Folgenden eine Testreihe durchgeführt. Zunächst wurde die SID des aktuell angemeldeten Benutzers mit dem Befehl \glqq \texttt{whoami /user}\grqq{} in der Eingabeaufforderung bestimmt, um im Tool den richtigen Benutzer auszuwählen. Da während der Installation und Konfiguration der virtuellen Maschine bereits ShellBag-Informationen entstanden sind, wurden zunächst die aktuellen ShellBag-Informationen des Benutzers im Tool ausgegeben und in der nachfolgenden Abbildung festgehalten. \\
\\
Basierend auf diesem Ergebnis wurde nun eine Testreihe aufgestellt und eine erneute Auswertung im Tool vergenommen. Die nachfolgende Tabelle \ref{akt} zeigt die durchgeführten Aktivitäten mit den entsprechenden Zeitangaben. Diese Zeiten wurden entsprechend in der Systemsteuerung eingestellt.

\begin{longtable}{|p{0.3\textwidth}|p{0.6\textwidth}|}
	\caption{Testreihe für Ordneraktivitäten} \label{akt} \vspace{1em} \\
	\hline
	\cellcolor{gray!25}\textbf{Zeit} & \cellcolor{gray!25}\textbf{Aktivität} \\
	\hline
	03.06.2020 14:00 Uhr & Anlegen eines Ordners \glqq test\grqq{} \\
	\hline
	03.06.2020 14:05 Uhr & Öffnen und Schließen des Ordners \glqq test\grqq{} \\
	\hline
	04.06.2020 15:00 Uhr & Anlegen eines Ordners \glqq studium\grqq{} \\
	\hline
	04.06.2020 15:10 Uhr & Öffnen und Schließen des Ordners \glqq studium\grqq{} \\
	\hline
	05.06.2020 16:00 Uhr & Anlegen eines Unterordners \glqq softwareprojekt\grqq{} \\
	\hline
	05.06.2020 16:25 Uhr & Öffnen und Schließen des Ordners \glqq softwareprojekt\grqq{} \\
	\hline
	06.06.2020 10:00 Uhr & Standardordner je nachdem (root und volume je) \\
	\hline
\end{longtable}
\vspace{1em}

%Auswertung, ist Unterordner unter Ordner?, stimmen die Informationen wie Zeitstempel etc --> alles abgleichen aus Tabellen, Fehler?, Screenshot zeigen, stimmt Zuordnung Value - Subkey?, kurzes Fazit

%Nun wurde die im Tool integrierte Export-Funktion getestet. Dafür wurden die aus der Testreihe ausgegeben ShellBag-Informationen exportiert. \\
%Test, die Infos zu exportieren mit Screenshot, wie vorgehen, welches Format
%\\
%Im letzten Schritt wurde die Lösch-Funktion des Tools getestet. Dafür wurden die in der Testreihe entstandenen ShellBag-Informationen gelöscht. Im Anschluss wurde der Registrierungs-Editor geöffnet, um zu überprüfen, ob die ShellBag-Informationen des angemeldeten Benutzers unter \texttt{HKU$\backslash$SID\_User$\backslash$Software$\backslash$Classes$\backslash$Local Settings$\backslash$Software$\backslash$Microsoft$\backslash$Windows$\backslash$Shell$\backslash$BagMRU} tatsächlich entfernt wurden.

%Test, die ShellBags zu löschen, wie vorgehen, evtl Screenshot der Registry