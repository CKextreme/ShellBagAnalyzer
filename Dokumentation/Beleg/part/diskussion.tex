\section{Diskussion}
\vspace{0.5cm}
Im Folgenden wird sich mit den Ergebnissen der Arbeit auseinandergesetzt. Es erfolgt eine Auswertung, wie gut das Tool den genannten Qualitätsanforderungen entspricht und welche Verbesserungen hätten vorgenommen werden können.
\subsection{Allgemeines}
\vspace{0.3cm}
Grundlage für die Implementierung des Tools stellt das .NET Framework dar. Dieses hat den Nachteil, dass es ausschließlich auf Windows-Betriebssystemen verfügbar ist und auch in Zukunft keine Weiterentwicklung erfolgen wird \cite{netfw,ende}. Vielmehr ist eine Zusammenführung mehrerer Produktlinien zu .NET 5.0 im November 2020 geplant \cite{ende}. Wenn für das Tool zukünftig noch eine Post-Mortem-Analyse implementiert werden würde, dann wäre es somit nur unter Windows lauffähig. Die quelloffene Entwicklungsplattform .NET Core, mit der ebenfalls Programme entwickelt und ausgeführt werden können, hätte den Vorteil gehabt, dass es neben Windows- auch auf Linux- und macOS-Betriebssystemen verfügbar gewesen wäre und somit die ShellBags offline auch auf diesen Systemen hätten untersucht werden können \cite{netcore}. Der Nachteil hierbei ist jedoch, dass der Windows Forms Designer, welcher der Erstellung der grafischen Benutzeroberfläche dient, bisher nur in einer Preview-Version vorliegt, was die Erstellung einer GUI erschweren würde \cite{preview}. So zum Beispiel wird das DataGridView-Steuerelement, welches der tabellarischen Anzeige von Daten dient, hier noch nicht unterstützt, was somit die Nutzung von .NET Core bisher nicht ermöglicht \cite{datagrid,dient}. Aus diesem Grund wurde auf das .NET Framework zurückgegriffen. Es ist jedoch denkbar, dass bei vollständiger Entwicklung eine Portierung vom .NET Framework zu .NET Core oder .NET 5.0 durchgeführt wird \cite{port}.

Weiterhin bleibt festzuhalten, dass die Auswertung der ShellBag-Informationen auf bereits geprüften Erkenntnissen basiert. So erfolgte eine Auswertung der File Entry-, Root Folder- sowie Volume Shell Items, deren Struktur bereits im Rahmen der Bachelorarbeit von Anna-Lena Totzauer untersucht wurde \cite{ba}. Darüber hinaus erlangte man aufgrund der vorliegenden ShellBag-Informationen von Conny Karras die Erkenntnis, dass ShellBags von sogenannten Shared Folders mit dem Class Type Indicator von 0xC3 unter \texttt{HKU$\backslash$SID\_User$\backslash$Software$\backslash$Micro- \newline soft$\backslash$Windows$\backslash$Shell$\backslash$BagMRU}, also in der NTUSER.dat abgelegt werden. Dieser Class Type Indicator wurde nach Joachim Metz auch schon gesichtet und in die Gruppe der Network Location Shell Items eingeordnet. Nach Metz existieren noch weitere Arten von Shell Items. Diese wurden jedoch bisher nicht analysiert, weshalb auf eine Einbeziehung in das selbst implementierte Tool aufgrund der fehlenden Überprüfung auf Korrektheit verzichtet wurde. Es ist jedoch durchaus denkbar, das Tool in Zukunft zu erweitern, sofern die Aussagen von Joachim Metz belegt werden können. Außerdem wurde davon ausgegangen, dass Root Folder Shell Items eine Größe von 20 Byte besitzen, um zu vermeiden, dass Shell Items mit dem Class Type Indicator von 0x1F mit einer anderen Struktur fehlerhaft der Kategorie der Root Folder Shell Items zugeordnet werden, obwohl dies gar nicht der Fall ist. \cite{shelltype}

Außerdem bleibt anzumerken, dass dieses Tool eine forensische Analyse der ShellBag-Infor- \newline mationen eines Windows 10 Betriebssystems ermöglicht. Wie bereits in der Bachelorarbeit von Anna-Lena Totzauer herausgefunden wurde, existieren durchaus Unterschiede zwischen den verschiedenen Betriebssystemen. Es ist somit denkbar, das Tool zukünftig auch für andere Windows-Betriebssysteme zu erweitern. \cite{ba,lo2014windows}

\subsection{Abgleich mit den Qualitätsanforderungen}
\vspace{0.3cm}
Nachfolgend wird diskutiert, ob das Tool die in Kapitel \ref{quali} genannten Qualitätsanforderungen erfüllen konnte.

\paragraph{Kompatibilität:}
Bezüglich der Kompatibilität des Tools konnten die Anforderungen erfüllt werden. Aufgrund der Nutzung des .NET Frameworks ist eine ShellBag-Live-Analyse ausschließlich auf Windows-Systemen, speziell unter Windows 10, möglich.

\paragraph{Integrität:}
Wie sich bei der Validierung des Tools herausstellte, wurden alle ShellBag-Einträge aus der Registry korrekt und vollständig in das Tool übertragen, weshalb hierbei eine Integrität sichergestellt werden kann. Auch die hierarchische Struktur der Einträge konnte beibehalten werden. Im Exportmodus blieb die hierarchische Struktur der ShellBags ebenfalls erhalten und es erfolgten zudem auch keine Manipulationen. Alle Einträge wurden vollständig übernommen, weshalb auch diese Anforderung erfüllt werden kann.

\paragraph{Zuverlässigkeit:}
Grundsätzlich arbeitet das Tool zuverlässig. Grundlegende Fehlermeldungen werden im Programm abgefangen. Ist beispielsweise ein Zeitstempel im Value eines File Entry Shell Items inkorrekt oder ungültig, so wird dieser Fehler abgefangen und dem Zeitstempel der Wert null zugewiesen. Somit wird vermieden, dass im Tool falsche Informationen ausgegeben werden. Ein eventueller Absturz des Tools kann jedoch nicht ausgeschlossen werden, da auch unvorhersehbare Fehler auftreten können, welche bisher nicht vom Programm abgefangen werden. In diesem Fall sollte eine Optimierung des Quellcodes stattfinden.

\paragraph{Benutzerfreundlichkeit:}
Das Tool ermöglicht mit Hilfe der Bedienungsanleitung einen leichten Einstieg für den Nutzer. Es wurde schlicht und benutzerfreundlich gestaltet. Alle Voraussetzungen zum Betrieb sowie Auswahlmöglichkeiten im Tool wurden in der Bedienungsanleitung erläutert.

\paragraph{Leistungseffizienz:}
Die Leistungseffizienz kann anhand der im Tool ausgegebenen Ladezeit bewertet werden. Die Zeit beschreibt die Dauer, welche das Tool benötigte, um die ShellBag-Informationen aus der Registry auszuwerten. Diese Zeit führte zu einem zufriedenstellenden Ergebnis, denn 7 ShellBag-Einträge konnten innerhalb von 20 ms ausgewertet werden, das entspricht im Schnitt 2,86 ms pro ShellBag-Eintrag. Bei Tests im Verlauf der Arbeit konnten teilweise sogar noch geringere Ladezeiten pro ShellBag-Eintrag festgestellt werden. Dies ist immer abhängig davon, wie viele Prozesse parallel laufen.

\paragraph{Portabilität:}
Wie in der Validierungsphase festgestellt werden konnte, war eine schnelle Portierung zwischen dem Host- und Gastsystem in einer VM möglich. Das Tool war sofort einsatzbereit. Die einzige Voraussetzung ist die Installation des .NET Frameworks 4.8.

