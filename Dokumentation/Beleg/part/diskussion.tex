\section{Diskussion}
Im Folgenden wird sich mit den Ergebnissen der Arbeit auseinandergesetzt. Es erfolgt eine Auswertung, wie gut das Tool den genannten Qualitätsanforderungen entspricht und welche Verbesserungen hätten vorgenommen werden können.

Grundlage für die Implementierung des Tools stellt das .NET Framework dar. Dieses hat den Nachteil, dass es ausschließlich auf Windows-Betriebssystemen verfügbar ist \cite{netfw} und auch in Zukunft keine Weiterentwicklung erfolgen wird \cite{ende}. Vielmehr ist eine Zusammenführung mehrerer Produktlinien zu .NET 5.0 im November 2020 geplant \cite{ende}. Wie bereits in Kapitel 3 erläutert, ist das Tool somit bei einer Post-Mortem-Analyse der ShellBags nur unter Windows möglich. Die quelloffene Entwicklungsplattform .NET Core, mit der ebenfalls Programme entwickelt und ausgeführt werden können, hätte den Vorteil gehabt, dass es neben Windows- auch auf Linux- und macOS-Betriebssystemen verfügbar gewesen wäre \cite{netcore}. Der Nachteil hierbei ist jedoch, dass der Windows Forms Designer, welcher der Erstellung der grafischen Benutzeroberfläche dient, bisher nur in einer Preview-Version vorliegt, was die Erstellung einer GUI erschweren würde \cite{preview}. Aus diesem Grund wurde auf das .NET Framework zurückgegriffen. Es ist jedoch denkbar, dass bei vollständiger Entwicklung eine Portierung vom .NET Framework zu .NET Core durchgeführt wird \cite{port}.
%Linux etc vllt irrelevant, da nur Onlineanalyse betrachtet

Weiterhin bleibt festzuhalten, dass die Auswertung der ShellBag-Informationen auf bereits geprüften Erkenntnissen basiert. So erfolgte eine Auswertung der File Entry-, Root Folder- sowie Volume Shell Items, dessen Struktur bereits im Rahmen der Bachelorarbeit von Anna-Lena Totzauer untersucht wurde \cite{ba}. Darüber hinaus gelang man aufgrund der vorliegenden ShellBag-Informationen von Conny Karras zu der Erkenntnis, dass ShellBags von sogenannten Shared Folders mit dem Class Type Indicator von 0xC3 unter \texttt{HKU$\backslash$SID\_User$\backslash$Software$\backslash$Micro- \newline soft$\backslash$Windows$\backslash$Shell$\backslash$BagMRU}, also in der NTUSER.dat abgelegt werden. Dieser Class Type Indicator wurde nach Joachim Metz auch schon gesichtet und in die Gruppe der Network Location Shell Items eingeordnet. Nach Metz existieren noch weitere Arten von Shell Items. Diese konnten jedoch bisher nicht beobachtet werden, weshalb auf eine Einbeziehung in das selbst implementierte Tool aufgrund der fehlenden Überprüfung auf Korrektheit verzichtet wurde. Es ist jedoch durchaus denkbar, das Tool in Zukunft zu erweitern, sofern die Aussagen von Joachim Metz belegt werden können. \cite{shelltype}

%Abgleich mit Qualitätsanforderungen, alles erfüllt?
%wo ist Verbesserung notwendig?
