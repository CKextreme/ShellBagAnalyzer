% This is the main tex file
% Please leave the following settings unchanged!
% ___________________________________________________________________
\documentclass{scrartcl}%


\usepackage{amsmath}%
\usepackage{amsfonts}%
\usepackage{amssymb}%
\usepackage{graphicx}
\usepackage{geometry}
\usepackage{hyperref}
\usepackage{color}
\usepackage{caption}
\usepackage{ngerman} 
\usepackage[utf8]{inputenc}
\usepackage{tabularx}
\usepackage{longtable}
\usepackage{colortbl}
\usepackage{float}
\usepackage{xcolor}
\usepackage[onehalfspacing]{setspace}
\usepackage{graphicx}
\usepackage{url}


%  Definition of some mathematical environments
\newtheorem{theorem}{Theorem}
\newtheorem{acknowledgement}[theorem]{Acknowledgement}
\newtheorem{algorithm}[theorem]{Algorithm}
\newtheorem{axiom}[theorem]{Axiom}
\newtheorem{case}[theorem]{Case}
\newtheorem{claim}[theorem]{Claim}
\newtheorem{conclusion}[theorem]{Conclusion}
\newtheorem{condition}[theorem]{Condition}
\newtheorem{conjecture}[theorem]{Conjecture}
\newtheorem{corollary}[theorem]{Corollary}
\newtheorem{criterion}[theorem]{Criterion}
\newtheorem{definition}[theorem]{Definition}
\newtheorem{example}[theorem]{Example}
\newtheorem{exercise}[theorem]{Exercise}
\newtheorem{lemma}[theorem]{Lemma}
\newtheorem{notation}[theorem]{Notation}
\newtheorem{problem}[theorem]{Problem}
\newtheorem{proposition}[theorem]{Proposition}
\newtheorem{remark}[theorem]{Remark}
\newtheorem{solution}[theorem]{Solution}
\newtheorem{summary}[theorem]{Summary}
\newenvironment{proof}[1][Proof]{\textbf{#1.} }{\ \rule{0.5em}{0.5em}}

\parindent 0.0mm
\geometry{a4paper,left=30mm,right=25mm, top=25mm, bottom=2cm}

% ___________________________________________________________________
% ___________________________________________________________________


\begin{document}
	\title{Softwarepraktikum: \\ 
		Implementierung eines Tools zur forensischen Analyse von ShellBags unter Windows 10}
	\author{Conny Karras und Anna-Lena Totzauer
		\\
		\\ \small{Cybercrime/Cybersecurity, CY19wC-M}
		\\ \small{Betreuer: Prof. Dr. rer. nat. Labudde}
		\\ \small{Bearbeitungszeitraum: April 2020 - Juli 2020}
		\\ \footnotesize{}}

   	\date{}
	\maketitle
 	\thispagestyle{empty}

	 \newpage 
	 \tableofcontents
	 \thispagestyle{empty}
	
	 \newpage
	 \pagenumbering{Roman}
	 \listoffigures
	 \addcontentsline{toc}{section}{Abbildungsverzeichnis}
	 
	 \listoftables
	 \addcontentsline{toc}{section}{Tabellenverzeichnis}
	 \newpage
	 
	 \pagenumbering{arabic}

	
	
	% Aufgabenstellung und Motivation
	\section{Aufgabenstellung und Motivation}
\vspace{0.5cm}
Dr. Edmund Locards Austauschprinzip \glqq Every contact leaves a trace\grqq{} wurde ursprünglich für die reale Welt formuliert. Doch auch in der digitalen Welt hinterlassen Aktivitäten Spuren auf einem System. Hierbei werden unter anderem Einträge in der Windows-Registry vorgenommen. \cite[S.5]{carvey2011windows} 

Die sogenannten ShellBags sind ein Teil dieser Registrierungsdatenbank. Sie speichern zum einen Einstellungen wie Fenstergröße, -position und Ansichtseinstellungen, zum anderen wird ein Eintrag abgelegt, sobald ein Ordner erstellt und geöffnet wurde. \cite[S.26]{kavrestad2018fundamentals} 

Welche Informationen die ShellBag-Registrierungsschlüssel speichern und wie sie aufgebaut und im Klartext zu verstehen sind, wurde bereits in der Bachelorarbeit von Anna-Lena Totzauer zum Thema \glqq Forensische Analyse von Windows ShellBags\grqq{} genauer untersucht. Basierend auf dieser Arbeit soll nun im Rahmen eines Softwarepraktikums ein Tool entwickelt werden, welches eine forensische Live-Analyse von ShellBags unter Windows 10 ermöglicht. Dafür sollen die in der Registry enthaltenen Informationen in einer für den Menschen unmittelbar lesbaren Form dargestellt werden. Darüber hinaus soll es möglich sein, die gefundenen Einträge zu exportieren. Mithilfe dieses Tools soll die Analyse der ShellBags erleichtert werden, da sie so nicht mehr händisch ausgelesen werden müssen. Dies führt auch zu einer Vermeidung von menschlichen Fehlern.  Darüber hinaus können die gewonnenen Informationen durch einen Export gesichert werden. \newline
Nachfolgend werden zunächst die theoretischen Grundlagen für ShellBags unter Windows 10 aufgegriffen. Darauf folgen die Qualitätsanforderungen an das Tool, Informationen zur praktischen Umsetzung, eine Bedienungsanleitung sowie die Validierung des Tools. Im Anschluss erfolgt eine Diskussion darüber, ob das Tool die zuvor festgelegten Qualitätsanforderungen erfüllt und was gegebenenfalls hätte optimiert werden können. Die beiden darauffolgenden Kapitel beinhalten ein zusammenfassendes Fazit sowie einen Ausblick. 


	%Zielsetzung, Motivation
	
	% Grundlagen ShellBags
	\section{Theoretische Grundlagen}
Nachfolgend werden die Grundlagen für ShellBags erläutert, welche für die Implementierung des Tools notwendig sind. Es handelt sich somit nur um einen Auszug der für das Projekt wichtigsten Grundlagen. 

\subsection{Allgemeines}
Die ShellBag-Schlüssel sind ein Teil der Windows-Registrierungsdatenbank, kurz Registry genannt, welche Konfigurationsdaten des Betriebssystems und seinen Programmen sowie Benutzereinstellungen speichert \cite[S.215]{anson2012mastering}. Die Registry besteht aus fünf Root Keys, von denen zwei als Master Keys bezeichnet werden, die tatsächlich physisch in der Registry abgelegt sind. Die anderen drei Keys sind nur abgeleitet von Schlüsseln unter den beiden Master Keys und sind somit Spiegelungen. \cite[S.219]{anson2012mastering} 

Die fünf Root Keys werden nachfolgend erläutert:
\begin{itemize}
	\item \texttt{HKEY\_LOCAL\_MACHINE} ist ein Master Key und wird somit nicht von anderen Keys abgeleitet. Dieser Key enthält Konfigurationsdaten des Computers.
	\item \texttt{HKEY\_USERS} ist ebenfalls ein Master Key. Dieser Key enthält Informationen über alle Benutzer, die jemals am System eingeloggt waren.
	\item  \texttt{HKEY\_CURRENT\_USER} ist abgeleitet von einem Schlüssel unter HKU. Er enthält die Einstellungen des aktuell angemeldeten Benutzers.
	\item \texttt{HKEY\_CLASSES\_ROOT} ist dafür zuständig, dass zu einem bestimmten Dateityp das richtige Programm ausgewählt wird. Dieser Schlüssel wird abgeleitet aus HKLM und HKCU.
	\item \texttt{HKEY\_CURRENT\_CONFIG} enthält Konfigurationseinstellungen der Hardware und wird abgeleitet von HKLM. \cite[S.219]{anson2012mastering}
\end{itemize}
Im Registrierungs-Editor werden die Keys genau in dieser Form dargestellt, allerdings werden die Informationen physisch gesehen in verschiedene Dateien, die sogenannten Hives, ausgelagert. Die Informationen werden somit nicht so zentral und zusammenhängend wie im Registrierungs-Editor, sondern über die gesamte Festplatte verteilt abgelegt. \cite[S.220]{anson2012mastering} 

Wo diese Hives tatsächlich auf der Festplatte abgelegt sind, erfährt man unter \texttt{HKEY\_LOCAL\_MA- \newline CHINE$\backslash$SYSTEM$\backslash$CurrentControlSet$\backslash$Control$\backslash$hivelist}. Wenn also das System die Hives in die Registry laden will, wird in diesem Key geprüft, wo die Dateien abgelegt sind. \cite[S.225]{anson2012mastering} \\
\\
Unter Windows 10 werden die ShellBag-Schlüssel unter den folgenden nutzerspezifischen Hives abgelegt: 
\begin{itemize}
	\item \texttt{NTUSER.dat$\backslash$Software$\backslash$Microsoft$\backslash$Windows$\backslash$Shell$\backslash$BagMRU}
	\item \texttt{NTUSER.dat$\backslash$Software$\backslash$Microsoft$\backslash$Windows$\backslash$Shell$\backslash$Bags}
	\item \texttt{USRCLASS.dat$\backslash$Local Settings$\backslash$Software$\backslash$Microsoft$\backslash$Windows$\backslash$Shell$\backslash$ \newline BagMRU}
	\item \texttt{USRCLASS.dat$\backslash$Local Settings$\backslash$Software$\backslash$Microsoft$\backslash$Windows$\backslash$Shell$\backslash$ \newline Bags} \cite[S.26]{kavrestad2018fundamentals}
\end{itemize}
Im Registrierungs-Editor befinden sich diese Informationen unter \texttt{HKU$\backslash$SID\_User$\backslash$Software$\backslash$ \newline Microsoft$\backslash$Windows$\backslash$Shell} bzw. unter\ \  \texttt{HKU$\backslash$SID\_User$\backslash$Software$\backslash$Classes$\backslash$Local Settings$\backslash$ \newline Software$\backslash$Microsoft$\backslash$Windows$\backslash$Shell}. \newline
Die beiden wichtigsten Schlüssel in den ShellBags sind \glqq BagMRU\grqq{} und \glqq Bags\grqq{}. Während \glqq Bags\grqq{} vom Benutzer definierte Einstellungen wie Fenstergröße, -position und Art der Ansicht speichert, enthält der \glqq BagMRU\grqq{}-Key die Verzeichnisstruktur mit den dazugehörigen Ordnernamen. \cite{lo2014windows} 

Der \glqq BagMRU\grqq{}-Key ist somit vor allem für forensische Analysen besonders relevant. In der Bachelorarbeit von Anna-Lena Totzauer konnte herausgefunden, werden, dass die wichtigsten Informationen bezüglich Ordneraktivitäten unter \texttt{HKU$\backslash$SID\_User$\backslash$Software$\backslash$Classes$\backslash$Local Settings$\backslash$Software$\backslash$Microsoft$\backslash$Windows$\backslash$Shell$\backslash$BagMRU} zu finden sind.  \cite{ba} \newline
Nach Joachim Metz existieren verschiedene Arten von Shell Items, die aufgrund von Ordnerakivitäten angelegt werden. So gibt es die File Entry Shell Items, welche vom Benutzer angelegte Ordner bzw. ZIP-komprimierte Ordner repräsentieren sowie Root Folder- und Volume Shell Items, welche jeweils standardmäßig vorhandene Ordner darstellen. Um welche Art von Shell Item es sich handelt, bestimmt der Class Type Indicator im jeweiligen Value. \cite{shelltype} 

Anhand der ShellBag-Informationen von Conny Karras konnte eine weitere Erkenntnis gewonnen werden. Hier konnten neben File Entry-, Root Folder- und Volume Shell Items in der USRCLASS.dat auch Informationen in der NTUSER.dat festgestellt werden. Die nachfolgende Abbildung \ref{img:net} zeigt einen entstandenen Value.

\begin{figure}[H]
	\centering
	\includegraphics[width=0.8\textwidth]{part/_images/shared.png}
	\caption{Value aus der NTUSER.dat, der das HSMW-Laufwerk repräsentiert} 
	\label{img:net}
\end{figure}

Die Values der in der NTUSER.dat gefundenen ShellBag-Einträge hatten den Class Type Indicator 0xC3. Dieser Indicator konnte ebenfalls von Joachim Metz bereits beobachtet werden. Er ordnete diese Art in die Gruppe der Network Location Shell Items ein. Der Name, der im Value dargestellt wird, zeigt, dass es sich bei diesen Shell Items um sogenannte Shared Folders, also freigegebene Ordner handelt. Eine solche Ordnerfreigabe ist auch an der Hochschule Mittweida möglich, um über VPN auch von außerhalb der Hochschule beispielsweise auf den Windows-Home-Bereich zugreifen zu können. Da also Shell Items auch in der NTUSER.dat gespeichert werden können, ist somit für die Analyse im Tool auch der Pfad \texttt{HKU$\backslash$SID\_User$\backslash$Software$\backslash$Microsoft$\backslash$Windows$\backslash$Shell$\backslash$BagMRU} relevant \cite{shelltype,hsmw}
%Sofern auf eine Dateifreigabe im Netzwerk eine permanente Verbindung eingerichtet wird, entsteht ein Netzlaufwerk, das als virtuelles Laufwerk die Ordner und Dateien eines Servers auf dem Client wie gewohnt anzeigt. Alternativ können Dateifreigaben auch ohne Netzlaufwerk benutzt werden, indem Verzeichnisse oder Dateien über die Uniform Naming Convention, d. h. in der Form \\servername\freigabename\dateiname, direkt angesprochen werden.

Die Kategorisierung der verschiedenen Shell Items muss bei der Implementierung des Tools beachtet werden, um die Einträge voneinander unterscheiden zu können, da die Shell Items unterschiedliche Strukturen aufweisen.

\subsection{Beispiel: Value eines File Entry Shell Items}
Für die Visualisierung der ShellBag-Informationen im selbst implementierten  Tool sind die Pfade \texttt{HKU$\backslash$SID\_User$\backslash$Software$\backslash$Classes$\backslash$Local Settings$\backslash$Software$\backslash$Microsoft$\backslash$Windows$\backslash$ \newline Shell$\backslash$BagMRU} und \texttt{HKU$\backslash$SID\_User$\backslash$Software$\backslash$Microsoft$\backslash$Windows$\backslash$Shell$\backslash$BagMRU} für die Ordneraktivitäten aller angemeldeten Benutzer besonders relevant. Hier müssen die entstandenen Values berücksichtigt werden, die verschiedene Arten von Shell Items repräsentieren. \newline
Die nachfolgende Abbildung \ref{img:value} zeigt beispielhaft den Value eines File Entry Shell Items, welcher sich unter dem zuvor genannten Pfad befindet. Je nachdem, um welche Art von Shell Item es sich handelt, müssen die verschiedenen Strukturen berücksichtigt werden. So weisen Root Folder bzw. Volume Shell Items eine andere Struktur in den Values auf als der hier abgebildete Value eines File Entry Shell Items.

\begin{figure}[H]
	\centering
	\includegraphics[width=0.8\textwidth]{part/_images/value.png}
	\caption{Ausschnitt des Values eines File Entry Shell Items, adaptiert nach \cite{lo2014windows}} 
	\label{img:value}
\end{figure}
Die hier abgebildeten Werte müssen im Little-Endian-Format interpretiert werden. Der Class Type Indicator an Offset 0x02 beträgt 0x31 und steht für einen vom Benutzer angelegten Ordner. Wie zu erkennen ist, sind alle drei Zeitstempel identisch. Im Rahmen der Bachelorarbeit wurde herausgefunden, dass es sich bei den MS-DOS Date and Time Zeitstempeln um die UTC-Zeit handelt, an dem der Ordner angelegt wurde. Somit ist die Creation Date and Time die einzig korrekte Zeit im Value. Die Ursache dafür liegt in der Master File Table des NTFS-Dateisystems, aus welcher die Zeitstempel stammen. \cite{ba} 

\subsection{Typen von Shell Items}
Wie zuvor erwähnt, existieren verschiedene Arten von Shell Items, welche durch den Class Type Indicator im jeweiligen Value definiert werden. Dieser sollte somit bei der Auswertung im implementierten Tool zuerst analysiert werden, um die Struktur des Values bestimmen zu können. 

Die nachfolgenden Tabellen zeigen die Informationen der verschiedenen Shell Items, die im implementierten Tool dargstellt werden sollen. Wichtig ist, dass die Informationen im Little-Endian-Format gelesen werden müssen. Die Offsets sind im Hexadezimalformat dargestellt. \newline
Die wichtigsten Informationen der \textbf{Root Folder Shell Items} sind in Tabelle \ref{rfs} dargestellt.

\begin{longtable}{|p{0.2\textwidth}|p{0.2\textwidth}|p{0.5\textwidth}|}
	\caption{Aufbau des Values eines Root Folder Shell Items, adaptiert nach \cite{shelltype}} \label{rfs} \vspace{1em} \\
	\hline
	\cellcolor{gray!25}\textbf{Offset} & \cellcolor{gray!25}\textbf{Größe in Byte} & \cellcolor{gray!25}\textbf{Beschreibung} \\
	\hline
    0x00 & 2 & Größe ShellBag-Eintrag\\
	\hline
	0x02 & 1 & Class Type Indicator = 0x1F \\
	\hline
	0x03 & 1 & Sort Index (Spezifizierung des Typs, siehe Tabelle) \\
	\hline
	0x04 & 16 & GUID der Form XXXXXXXX-XXXX-XXXX-XXXX-XXXXXXXXXXXX, eindeutig für jeden Typ (8 Byte von links im Little-Endian-Format lesen, 8 Byte von rechts im Big-Endian-Format lesen \cite{ba}) \\
	\hline
\end{longtable}
\vspace{1em}

Die Sort Indizes der Root Folder Shell Items sind in Tabelle \ref{sort} aufgelistet. 

\begin{longtable}{|p{0.2\textwidth}|p{0.3\textwidth}|}
	\caption{Sort Indizes, adaptiert nach \cite{shelltype}} \label{sort} \vspace{1em} \\
	\hline
	\cellcolor{gray!25}\textbf{Sort Index} & \cellcolor{gray!25}\textbf{Beschreibung} \\
	\hline
	0x00, 0x68 & Internet Explorer\\
	\hline
	0x42 & Libraries \\
	\hline
	0x44 & Users \\
	\hline
	0x48 & My Documents \\
	\hline
	0x50 & My Computer \\
	\hline
	0x58 & My Network Places/ Network \\
	\hline
	0x60 & Recycle Bin \\
	\hline
	0x80 & My Games \\
	\hline
\end{longtable}
\vspace{1em}

Die wichtigsten Informationen der \textbf{Volume Shell Items} sind in Tabelle \ref{volume} dargestellt.

\begin{longtable}{|p{0.2\textwidth}|p{0.2\textwidth}|p{0.5\textwidth}|}
	\caption{Aufbau des Values eines Volume Shell Items, adaptiert nach \cite{shelltype}} \label{volume} \vspace{1em} \\
	\hline
	\cellcolor{gray!25}\textbf{Offset} & \cellcolor{gray!25}\textbf{Größe in Byte} & \cellcolor{gray!25}\textbf{Beschreibung} \\
	\hline
	0x00 & 2 & Größe ShellBag-Eintrag\\
	\hline
	0x02 & 1 & Class Type Indicator = 0x20-0x2F \\
	\hline
\end{longtable}
\vspace{1em}
%\textbf{Conny: Vllt könntest du auch nochmal auf der Seite von Metz schauen nach dem Class Type Indicator von Volume Shell Items, iwie verstehe ich den Text nicht mit bitmask???, ich denke Indikator 20-2f, Netzlaufwerk hatte bei mir Indicator von 0x2F - wahrscheinlich ist das auch ein Volume shell item mit Laufwerksbuchstabe an Offset 0x03 für 3 Byte, siehe BA Präsentation + welchen Inidcator zeigt es dir bei USB-Sticks an? bei mir 1F}
%Nach der Bachelorarbeit von Anna-Lena Totzauer besitzen Netzlaufwerke als Volume Shell Items einen Class Type Indicator von 0x2F. Der Laufwerksbuchstabe befindet sich an Offset 0x03 für 3 Byte. \cite{ba} \newline
Die wichtigsten Informationen der \textbf{File Entry Shell Items} sind in Tabelle \ref{fes} dargestellt.

\begin{longtable}{|p{0.2\textwidth}|p{0.2\textwidth}|p{0.5\textwidth}|}
	\caption{Aufbau des Values eines File Entry Shell Items, adaptiert nach \cite{shelltype}} \label{fes} \vspace{1em} \\
	\hline
	\cellcolor{gray!25}\textbf{Offset} & \cellcolor{gray!25}\textbf{Größe in Byte} & \cellcolor{gray!25}\textbf{Beschreibung} \\
	\hline
	0x00 & 2 & Größe ShellBag-Eintrag\\
	\hline
	0x02 & 1 & Class Type Indicator = 0x31 für Ordner, = 0x32 für ZIP-komprimierte Ordner \\
	\hline
	0x08 & 4 & Modification Date and Time (entspricht Creation Date and Time) \\
	\hline
	0x0E & variabel & Primary Name (kurzer Name) \\
	\hline
	variabel & variabel & Beginn Extension Block (die folgenden Offsets verstehen sich innerhalb des Extension Blocks)\\
	\hline
	0x00 & 2 & Größe Extension Block \\
	\hline
	0x02 & 2 & Extension Version (0009 steht für Windows 8.1 oder Windows 10) \\
	\hline
	0x04 & 4 & Extension Block Signatur (0xBEEF0004) \\
	\hline
	0x08 & 4 & Creation Date and Time im Format MS-DOS Date and Time \\
	\hline
	0x0C & 4 & Last Access Date and Time (entspricht Creation Date and Time)\\
	\hline
	variabel & variabel & Secondary Name (langer Name, außerhalb des Extension Blocks)\\
	\hline
\end{longtable}
\vspace{1em}
Die wichtigsten Informationen der \textbf{Network Location Shell Items} sind in Tabelle \ref{netw} aufgelistet.

\begin{longtable}{|p{0.2\textwidth}|p{0.2\textwidth}|p{0.5\textwidth}|}
	\caption{Aufbau des Values eines Network Location Shell Items, adaptiert nach \cite{shelltype}} \label{netw} \vspace{1em} \\
	\hline
	\cellcolor{gray!25}\textbf{Offset} & \cellcolor{gray!25}\textbf{Größe in Byte} & \cellcolor{gray!25}\textbf{Beschreibung} \\
	\hline
	0x00 & 2 & Größe ShellBag-Eintrag\\
	\hline
	0x02 & 1 & Class Type Indicator = 0xC3 \\
	\hline
	0x05 & variabel & Ort des freigegebenen Ordners \\
	\hline
\end{longtable}
\vspace{1em}






	%Shellbags Eintrag, Speicherort,...
	
	%Qualitätsanforderungen
	\section{Qualitätsanforderungen} \label{quali}
\vspace{0.5cm}
Nachfolgend werden die Qualitätsanforderungen an das Tool erläutert, welche bei der Implementierung berücksichtigt werden müssen, um eine optimale Eignung gewährleisten zu können.
\paragraph{Kompatibilität:}
Da die ShellBags in ihrer Form als Teil der Registry nur für Windows-Betriebssysteme existieren, kann die ShellBag-Live-Analyse auch nur auf Windows-Systemen erfolgen. Das implementierte Tool wird auch aufgrund der Nutzung des .NET Frameworks nur unter diesem Betriebssystem lauffähig sein \cite{netfw}.
\paragraph{Integrität:}
Im lesenden Modus der ShellBags aus der Windows-Registry dürfen Einträge nicht manipuliert, gelöscht oder eingefügt werden. Im Exportmodus sollte die hierarchische Struktur der ShellBags erkennbar sein und auch hier dürfen keine Manipulationen erfolgen.
\paragraph{Zuverlässigkeit:}
Aus forensischer Sicht soll das Tool auch bei fehlerhaften Einträgen eine zuverlässige Analyse und einen gewissen Teil-Output ermöglichen.
\paragraph{Benutzerfreundlichkeit:}
Das Programm sollte einfach gestaltet sein und auch Menschen mit geringen Kenntnissen über ShellBags einen leichten Einstieg ermöglichen.
\paragraph{Leistungseffizienz:}
Das Tool soll seine Ergebnisse nach einem angemessenen Zeitrahmen präsentieren können.
\paragraph{Portabilität:}
Es soll gewährleistet werden, dass die Software schnell zwischen mehreren Rechnern portiert werden kann, ohne dass dafür spezielle Installationen notwendig sind.








	% Implementierung
	\section{Praktische Umsetzung}
\vspace{0.5cm}
Im Anschluss erfolgt eine Beschreibung, wie bei der praktischen Umsetzung des Tools vorgegangen wurde. \\
\\
Grundlage für die Implementierung des Tools soll die integrierte Entwicklungsumgebung (IDE) Visual Studio 2019 von Microsoft darstellen \cite{vs}. Als Programmiersprache wird C\# mit dem .NET Framework verwendet. Bei dem .NET Framework, welches ein Teil der Microsoft Software-Plattform .NET ist, handelt es sich um eine umfangreiche Infrastruktur, in welcher Anwendungen programmiert, kompiliert, ausgeführt und verteilt werden können \cite[S.68]{bayer2008visual}. Dieses Framework ist ausschließlich auf Windows-Systemen verfügbar, das heißt, Anwendungen können nur unter Windows entwickelt und ausgeführt werden \cite{netfw}. \\
\\
Im Anschluss erfolgte die Implementierung der Live-Analyse der ShellBags. Im ersten Schritt wurde sich neben dem Aufbau der grafischen Benutzeroberfläche unter \textit{ShellBag.GUI} Zugang zu den ShellBag-Schlüsseln im Registrierungs-Editor unter \texttt{HKU$\backslash$SID\_User$\backslash$Software$\backslash$Classes$\backslash$ \newline Local Settings$\backslash$Software$\backslash$Microsoft$\backslash$Windows$\backslash$Shell$\backslash$BagMRU} bzw. unter \texttt{HKU$\backslash$SID\_User$\backslash$ \newline Software$\backslash$Microsoft$\backslash$Windows$\backslash$Shell} verschafft. Dies erfolgte in der Klasse \textit{ShellBag.Library. \newline ShellBagParser.cs}. Es wurde eine Auswahlmöglichkeit geschaffen, bei der man eine bestimmte SID wählen kann, dessen ShellBag-Schlüssel nun in das Tool eingelesen werden können. Darüber hinaus wurde die Beziehung zwischen der SID eines Benutzers und dem jeweiligen Benutzernamen identifiziert. Hintergrund für diesen Zusammenhang stellt zum einen der Befehl \glqq \texttt{whoami /user}\grqq{} dar, welcher die SID des aktuellen Benutzers ausgibt \cite{whoami}. Zum anderen gibt der Befehl \glqq \texttt{wmic useraccount where sid='<SID>' get name}\grqq{} den zur SID dazugehörigen Benutzernamen aus \cite{sid}. Diese Hilfsmethoden sind unter \textit{ShellBag.Library.Shell \newline Bags.ShellBagHelper.cs} festgeschrieben. Das Ergebnis des ersten Schrittes ist eine grafische Benutzeroberfläche, auf welcher die ShellBag-Schlüssel eines konkreten Benutzers aus dem Registrierungs-Editor eingelesen werden können. Die Darstellung der Informationen entspricht der Form, wie die Schlüssel im Registrierungs-Editor abgebildet werden, nur dass im implementierten Tool neben anderen ausschließlich die ShellBag-Schlüssel abgebildet und zusätzlich die Bezeichnungen der jeweiligen Ordner dargestellt sind, um sich bereits hier einen ersten Überblick verschaffen zu können. Darüber hinaus wird berechnet, wie viele Subkeys jedes Shell Item besitzt. Es besteht somit ein optischer Bezug zum Registrierungs-Editor, jedoch werden noch Zusatzinformationen abgebildet. Es erfolgte außerdem eine Implementierung der Ladezeit sowie der Anzahl aller Shell Items, um später die Leistungseffizienz bewerten zu können.

Nun wurden die Values der Shell Items ausgewertet. Für jedes Shell Item wurde dafür eine eigene Klasse unter \textit{ShellBag.Library.ShellBags.ShellItems} generiert, in der die spezifischen Informationen ausgelesen wurden. Diese Values wurden transparent im Hintergrund ausgewertet. Es erfolgte zunächst eine Zuordnung des Values zum passenden Subkey unter BagMRU. Die Zuordnung basiert darauf, dass der Value durch dieselbe Zahl beschrieben wird wie der passende Subkey unter BagMRU. Um zunächst zwischen den Arten von Shell Items zu unterscheiden, wurde der Class Type Indicator an Offset 0x02 betrachtet. Dies war notwendig, da die Shell Items verschiedene Strukturen aufweisen, je nachdem, um welche Art es sich handelt. Lautet der Class Type Indicator zum Beispiel 0x31, so handelt es sich um ein File Entry Shell Item und die Auswertung der Informationen erfolgt gemäß Tabelle \ref{fes}. \\
\\
Anschließend wurde die Exportfunktion im Tool implementiert. Dazu wurde eine eigene Klasse unter \textit{ShellBag.Library.ShellBags.Logging.FileExport.cs} generiert. Hier erfolgte eine separate Behandlung der ShellBags der NTUSER.dat und der USRCLASS.dat, das heißt, es wurde so programmiert, dass für beide Hives eine separate Textdatei angelegt wird. Die Darstellung der ShellBag-Informationen erfolgte in einer hierarchischen Form, sodass auch hier deutlich erkennbar ist, welche Elemente Kinder vom übergeordneten Knoten sind und welche Elemente sich auf gleicher Ebene befinden. Der Name der jeweiligen Textdatei wurde wie folgt gewählt: \glqq Export\_Hive-Name\_Datum\_Zeit.txt\grqq{}. Somit ist klar erkennbar, aus welchem Hive die exportierten ShellBags stammen und zu welcher Zeit der Export erfolgte.




%vllt noch ergänzen, welcher Schritt in welcher Methode/ Klasse bzw. in welcher Datei der Code zu finden ist --> zB auch Umwandlung GUID oder Zeitstempel: mit welcher Methode?



	
	% Bedienungsanleitung
	\section{Bedienungsanleitung}
\vspace{0.5cm}
Voraussetzung für die Inbetriebnahme des Tools \glqq ShellBag Analyzer\grqq{} ist die Installation von Microsoft .NET Framework 4.8 \footnote{Download unter \url{https://www.chip.de/downloads/Microsoft-.NET-Framework-4.8_54812855.html}, zuletzt verfügbar am 15.06.2020}. Dies ist notwendig, um Anwendungen auszuführen, welche mit dem .NET Framework entwickelt wurden \cite{nett}. Mit dem Windows Update vom 10. Mai 2019 wird das .NET Framework 4.8 bereits mitgeliefert \cite{update}. Das Tool wird als ZIP-Datei zur Verfügung gestellt und muss zuvor entpackt werden. Anschließend kann die Datei \glqq ShellBag.GUI.exe\grqq{} ausgeführt werden. \\
\\
Falls ein Computer mehrere Nutzer enthält, so ist zunächst die SID des zu untersuchenden Benutzerkontos zu bestimmen. Bei der Analyse des aktuell angemeldeten Benutzers kann dessen SID mit dem Befehl \glqq \texttt{whoami /user}\grqq{} in der Eingabeaufforderung bestimmt werden. Dies ist notwendig, um den richtigen Benutzer im Tool oben links auszuwählen. Nach der Auswahl der jeweiligen SID erscheint der korrespondierende Benutzername sowie im unteren Bereich die Anzahl der ShellBags und die Ladezeit, die benötigt wurde, um die Informationen aus der Registry zu laden und auszuwerten. Nun kann im linken Bereich der Baum für die ShellBags der NTUSER.dat und  USRCLASS.dat aufgeklappt und ein Shell Item ausgewählt werden. Durch Klicken auf ein solches Item erscheinen im rechten Bereich die Art des Shell Items sowie die dazugehörigen Informationen. Diese werden tabellarisch aufgelistet. \\
\\
Um die Exportfunktion der ShellBags anzuwenden, ist in der obigen Menüleiste \glqq Datei - Exportieren - Textdatei (.txt)\grqq{} zu wählen. Die ShellBag-Informationen werden anschließend in Textdateien, getrennt voneinander für die NTUSER.dat und USRCLASS.dat, in dem Verzeichnis abgelegt, in dem sich die ausführbare Datei befindet. Diese Dateien können im Anschluss mit einem Editor geöffnet und eingesehen werden. \\
\\
Die Beendigung des Tools erfolgt entweder über die Menüleiste \glqq Datei - Beenden\grqq{} oder über das Kreuz-Symbol rechts oben. In beiden Fällen ist die Beendigung mit der Auswahl des Buttons \glqq Ja\grqq{} zu bestätigen. \\
\\
Die hier aufgeführten Informationen lassen sich ebenfalls im \glqq ShellBag Analyzer\grqq{} unter dem Menüpunkt \glqq Über\grqq{} einsehen. \\
\\
Die nachfolgende Abbildung \ref{img:analyzer} zeigt einen Ausschnitt des \glqq ShellBag Analyzers\grqq{}.

\begin{figure}[H]
	\centering
	\includegraphics[width=1\textwidth]{part/analyzer.png}
	\caption{Ausschnitt des \glqq ShellBag Analyzers\grqq{}} 
	\label{img:analyzer}
\end{figure}

	
	% Validierung
	\section{Validierung}
\vspace{0.5cm}
Um das implementierte Tool zu testen, wurde eine virtuelle Maschine verwendet. Hierfür war es zunächst notwendig, die Virtualisierungssoftware VirtualBox \footnote{Download unter \url{https://www.chip.de/downloads/VirtualBox_23814448.html}, zuletzt verfügbar am 16.06.2020}  herunterzuladen. Im Anschluss war es möglich, bei Microsoft einen bereits fertigen Evaluierungscomputer als virtuelle Maschine (VM) mit Windows 10 \footnote{Download unter \url{https://developer.microsoft.com/de-de/windows/downloads/virtual-machines/}, zuletzt verfügbar am 16.06.2020}  herunterzuladen. Diese VM konnte nach dem Entpacken problemlos in VirtualBox importiert und ausgeführt werden. Nach der Installation der Gasterweiterungen \footnote{Download unter \url{http://download.virtualbox.org/virtualbox/}, zuletzt verfügbar am 16.06.2020} wurde ein gemeinsamer Ordner festgelegt, mit dem ein Datenaustausch zwischen dem Gast- und Host-System möglich ist, um das selbst implementierte Tool auch in der VM starten zu können. \\
\\
Um zu testen, ob das Tool eine korrekte Auswertung der ShellBag-Informationen vornimmt, wurde im Folgenden eine Testreihe durchgeführt. Zunächst wurde die SID des aktuell angemeldeten Benutzers mit dem Befehl \glqq \texttt{whoami /user}\grqq{} in der Eingabeaufforderung bestimmt, um im Tool den richtigen Benutzer auszuwählen. Die SID des aktuell angemeldeten Benutzers \texttt{windev2004eval$\backslash$user} lautete \texttt{S-1-5-21-3234566417-3121902272-2793192418-1001}. 

Da während der Installation und Konfiguration der virtuellen Maschine bereits ShellBag-Informationen entstanden sind, wurden zunächst die aktuellen ShellBag-Informationen des Benutzers im Tool ausgegeben und in der nachfolgenden Abbildung \ref{img:vorher} festgehalten. 

\begin{figure}[H]
	\centering
	\includegraphics[width=0.8\textwidth]{part/vorher.png}
	\caption{ShellBag-Einträge vor der Testreihe} 
	\label{img:vorher}
\end{figure}

Basierend auf diesem Ergebnis wurde nun eine Testreihe aufgestellt und eine erneute Auswertung im Tool vorgenommen. Die nachfolgende Tabelle \ref{akt} zeigt die durchgeführten Aktivitäten mit den entsprechenden Zeitangaben. Diese Zeiten wurden entsprechend in der Systemsteuerung eingestellt.

\begin{longtable}{|p{0.3\textwidth}|p{0.6\textwidth}|}
	\caption{Testreihe für Ordneraktivitäten} \label{akt} \vspace{1em} \\
	\hline
	\cellcolor{gray!25}\textbf{Zeit} & \cellcolor{gray!25}\textbf{Aktivität} \\
	\hline
    20.05.2020 10:11 Uhr & Anlegen eines Ordners \glqq test\grqq{} auf dem Desktop \\
	\hline
	20.05.2020 10:12 Uhr & Öffnen und Schließen des Ordners \glqq test\grqq{} \\
	\hline
	20.05.2020 10:13 Uhr & Anlegen eines ZIP-komprimierten Ordners \glqq studium\grqq{} auf dem Desktop \\
	\hline
	20.05.2020 10:15 Uhr & Öffnen und Schließen des ZIP-komprimierten Ordners \glqq studium\grqq{} \\
	\hline
	20.05.2020 10:16 Uhr & Anlegen eines Unterordners \glqq softwareprojekt\grqq{} unter \glqq test\grqq{} \\
	\hline
	20.05.2020 10:18 Uhr & Öffnen und Schließen des Unterordners \glqq softwareprojekt\grqq{} \\
	\hline
	20.05.2020 10:19 Uhr & Öffnen des Laufwerkes C:$\backslash$ im Datei-Explorer unter \glqq My Computer\grqq{}\\
	\hline
\end{longtable}
\vspace{1em}

Zunächst wurde die Ausgabe im Tool mit der Ausgabe im Registrierungs-Editor verglichen, um zu überprüfen, ob die Struktur der ShellBags korrekt übernommen wurde. Die nachfolgenden Abbildungen \ref{img:regedit} und \ref{img:nachher} zeigen einen Ausschnitt. 

\begin{figure}[H]
	\centering
	\includegraphics[width=0.8\textwidth]{part/regedit.png}
	\caption{ShellBag-Baumstruktur im Registrierungs-Editor nach der Testreihe} 
	\label{img:regedit}
\end{figure}

\begin{figure}[H]
	\centering
	\includegraphics[width=0.8\textwidth]{part/nachher.png}
	\caption{ShellBag-Baumstruktur im Tool nach der Testreihe} 
	\label{img:nachher}
\end{figure}

Wie zu erkennen ist, sind die beiden Strukturen identisch. Die ShellBag-Struktur wurde somit korrekt im Tool abgebildet. 

Im Anschluss erfolgte ein Abgleich der Daten im rechten Fenster, in dem alle Informationen eines Values in einer lesbaren Form dargestellt werden. 

Der Ordner \glqq test\grqq{} wurde als Subkey unter BagMRU angelegt und entspricht einem File Entry Shell Item mit dem Class Type Indicator von 0x31. Dies wurde korrekt im rechten Fenster erfasst. Auch der Primary- und Secondary Name wurde korrekt abgebildet. Die Extension Version lautet 0x0009, welche für Windows 10 steht. Alle drei Zeitstempel wurden ebenfalls korrekt umgewandelt. Es bleibt anzumerken, dass es sich hierbei um die UTC-Zeit handelt. Da in der virtuellen Maschine als Zeitzone die Pacific Time (UTC-8:00) eingestellt ist, müssen zu den in Tabelle \ref{akt} abgebildeten Zeiten durch die aktuelle Sommerzeit sieben Stunden hinzuaddiert werden, um die koordinierte Weltzeit zu erhalten \cite{zz}. Auch diese Zeit wurde im Tool korrekt abgebildet. 

Der Subkey des ZIP-komprimierten Ordners \glqq studium\grqq{} wurde ebenfalls unter BagMRU angelegt. Auch hier wurden die Informationen korrekt dargestellt. Im Unterschied zu einem normalen Ordner besitzt der ZIP-komprimierte Ordner den Class Type Indicator von 0x32. Diese Unterscheidung wird vom Tool ebenfalls erfasst. 

Der Eintrag zum Unterordner \glqq softwareprojekt\grqq{} wurde korrekt als Subsubkey unter dem Subkey, der für den Ordner \glqq test\grqq{} steht, angelegt. Auch diese Informationen des File Entry Shell Items wurden korrekt dargestellt. Der Primary Name wurde im Long File Name Format \glqq SOFTWA\raisebox{-0.9ex}{\~{}}1\grqq{} gespeichert. Im Secondary Name ist der vollständige Name abgelegt.

Das Öffnen des Laufwerkes C:$\backslash$ führte in den ShellBags dazu, dass unter dem Subkey, der für \glqq My Computer\grqq{} steht, ein Subkey angelegt wurde. Das Laufwerk repräsentiert ein Volume Shell Item mit dem Class Type Indicator von 0x2F. Diese Information war im rechten Fenster sichtbar, genau wie der Laufwerksbuchstabe.

Der übergeordnete Knoten des Laufwerkes C:$\backslash$ steht für das Root Folder Shell Item \glqq My Computer\grqq{} mit dem Class Type Indicator von 0x1F und dem Sort Index von 0x50. Dieser Eintrag war bereits vor der Testreihe durch vorher getätigte Einstellungen vorhanden. Es konnte somit jedoch bestätigt werden, dass es sich tatsächlich um den standardmäßig vorhandenen Ordner handelt, da das Laufwerk geöffnet wurde, indem zuvor auf \glqq My Computer\grqq{} geklickt wurde. Die GUID des Root Folder Shell Items wurde ebenfalls korrekt dargestellt. Diese lautet \texttt{20d04fe0-3aea-1069-a2d8-08002b30309d}.

Es konnte somit festgestellt werden, dass die Baumstruktur korrekt aus der Registry übernommen wurde und alle Informationen der jeweiligen Shell Items korrekt abgebildet und interpretiert wurden. \\
\\
Außerdem kann die Ladezeit betrachtet werden. Diese gibt an, welche Zeitdauer das Tool benötigt hat, um die ShellBag-Einträge auszuwerten und darzustellen. In diesem Fall benötigte das Tool 20 ms für 7 ShellBag-Einträge. Dies entspricht im Durchschnitt ca. 2,86 ms pro Eintrag. \\
\\
Nun wurde die im Tool integrierte Export-Funktion getestet. Dafür wurden die im Tool ausgegebenen ShellBag-Einträge der Testreihe exportiert. Es wurde jeweils eine .TXT-Datei für ShellBags der NTUSER.dat und der USRCLASS.dat im Verzeichnis, in dem die ausführbare Datei liegt, gespeichert. Diese Dateien wurden mit einem Editor geöffnet. Wie bereits im Tool erkennbar war, gab es für die Testreihe keine Einträge in der NTUSER.dat. Aus diesem Grund ist die Datei leer. Die Datei der USRCLASS.dat beinhaltet jedoch alle ShellBag-Einträge, die im Tool aufgrund der Testreihe ausgegeben wurden. Auch hier wurden die Informationen vollständig und in einer hierarchischen Struktur abgebildet. Die nachfolgende Abbildung \ref{img:export} zeigt den Ausschnitt der Datei. 

\begin{figure}[H]
	\centering
	\includegraphics[width=1\textwidth]{part/export.png}
	\caption{Ausschnitt der Datei mit exportierten ShellBag-Einträgen der USRCLASS.dat} 
	\label{img:export}
\end{figure}



	
	% Diskussion
	\section{Diskussion}
Im Folgenden wird sich mit den Ergebnissen der Arbeit auseinandergesetzt. Es erfolgt eine Auswertung, wie gut das Tool den genannten Qualitätsanforderungen entspricht und welche Verbesserungen hätten vorgenommen werden können.

Grundlage für die Implementierung des Tools stellt das .NET Framework dar. Dieses hat den Nachteil, dass es ausschließlich auf Windows-Betriebssystemen verfügbar ist \cite{netfw} und auch in Zukunft keine Weiterentwicklung erfolgen wird \cite{ende}. Vielmehr ist eine Zusammenführung mehrerer Produktlinien zu .NET 5.0 im November 2020 geplant \cite{ende}. Wie bereits in Kapitel 3 erläutert, ist das Tool somit bei einer Post-Mortem-Analyse der ShellBags nur unter Windows möglich. Die quelloffene Entwicklungsplattform .NET Core, mit der ebenfalls Programme entwickelt und ausgeführt werden können, hätte den Vorteil gehabt, dass es neben Windows- auch auf Linux- und macOS-Betriebssystemen verfügbar gewesen wäre \cite{netcore}. Der Nachteil hierbei ist jedoch, dass der Windows Forms Designer, welcher der Erstellung der grafischen Benutzeroberfläche dient, bisher nur in einer Preview-Version vorliegt, was die Erstellung einer GUI erschweren würde \cite{preview}. Aus diesem Grund wurde auf das .NET Framework zurückgegriffen. Es ist jedoch denkbar, dass bei vollständiger Entwicklung eine Portierung vom .NET Framework zu .NET Core durchgeführt wird \cite{port}.
%Linux etc vllt irrelevant, da nur Onlineanalyse betrachtet

Weiterhin bleibt festzuhalten, dass die Auswertung der ShellBag-Informationen auf bereits geprüften Erkenntnissen basiert. So erfolgte eine Auswertung der File Entry-, Root Folder- sowie Volume Shell Items, dessen Struktur bereits im Rahmen der Bachelorarbeit von Anna-Lena Totzauer untersucht wurde \cite{ba}. Darüber hinaus gelang man aufgrund der vorliegenden ShellBag-Informationen von Conny Karras zu der Erkenntnis, dass ShellBags von sogenannten Shared Folders mit dem Class Type Indicator von 0xC3 unter \texttt{HKU$\backslash$SID\_User$\backslash$Software$\backslash$Micro- \newline soft$\backslash$Windows$\backslash$Shell$\backslash$BagMRU}, also in der NTUSER.dat abgelegt werden. Dieser Class Type Indicator wurde nach Joachim Metz auch schon gesichtet und in die Gruppe der Network Location Shell Items eingeordnet. Nach Metz existieren noch weitere Arten von Shell Items. Diese konnten jedoch bisher nicht beobachtet werden, weshalb auf eine Einbeziehung in das selbst implementierte Tool aufgrund der fehlenden Überprüfung auf Korrektheit verzichtet wurde. Es ist jedoch durchaus denkbar, das Tool in Zukunft zu erweitern, sofern die Aussagen von Joachim Metz belegt werden können. \cite{shelltype}

%Abgleich mit Qualitätsanforderungen, alles erfüllt?
%wo ist Verbesserung notwendig?

	
	% Ergebnisse und Diskussion
	\section{Fazit}
...
%durch .net framework nur für windows möglich

	% Ausblick
	\section{Ausblick}
\vspace{0.5cm}
In Zukunft wäre es denkbar, eine Portierung vom .NET Framework auf .NET Core oder .NET 5.0 vorzunehmen. Dies würde die Möglichkeit eröffnen, neben der Live-Analyse auch eine Post-Mortem-Analyse der Windows-ShellBags zu implementieren, deren Auswertung neben Windows- auch auf Linux- oder macOS-Betriebssystemen möglich wäre. Der Vorteil davon wäre somit, dass das Tool eine höhere Kompatibilität aufweist und auf verschiedenen Betriebssystemen lauffähig ist, je nachdem welches Betriebssystem die forensische Workstation besitzt. \cite{netcore,port} \\
\\
Weiterhin existieren neben File Entry-, Root Folder-, Volume- und Network Location Shell Items noch andere Arten. In der vorliegenden Arbeit wurden nur die bereits geprüften Erkenntnisse in der Auswertung der ShellBags berücksichtigt. Das Tool sollte daher zukünftig erweitert werden, sofern beispielweise die Informationen von Joachim Metz bezüglich Shell Items bestätigt werden. Auch vor allem die ShellBags in der NTUSER.dat haben noch ein großes Analysepotenzial. \cite{shelltype} \\
\\
Außerdem dient das Tool bisher einer forensischen Analyse der ShellBags eines Windows 10 Betriebssystems. Da jedoch Unterschiede zwischen den einzelnen Betriebssystemen existieren, sollte das Tool zukünftig auch um die anderen Betriebssystemversionen erweitert werden, um die Anwendungsfähigkeit des Tools zu erweitern. \cite{ba}


	\newpage
	% Literatur
	\bibliographystyle{alpha}
	\bibliography{Literatur}
	\addcontentsline{toc}{section}{Literaturverzeichnis}
	
	\newpage
	\thispagestyle{empty}
	\section*{Selbstständigkeitserklärung}
	\vspace{15mm}
	
	Hiermit erkläre ich, dass ich die vorliegende Arbeit selbstständig und nur unter Verwendung der angegebenen
	Quellen und Hilfsmittel angefertigt habe. Sämtliche Stellen der Arbeit, die im Wortlaut oder dem Sinn nach
	Publikationen oder Vorträgen anderer Autoren entnommen sind, habe ich als solche kenntlich gemacht. Diese
	Arbeit wurde in gleicher oder ähnlicher Form noch keiner anderen Prüfungsbehörde vorgelegt oder anderweitig
	veröffentlicht.
	
	\vspace{2cm}
	
	\begin{tabular}{lp{2em}l}
		\hspace{4cm} && \hspace{7cm} \\ \cline{1-1}\cline{3-3} \\
		Ort, Datum && Unterschrift
	\end{tabular}
	\vspace{1cm}

	\begin{tabular}{lp{2em}l}
	\hspace{4cm} && \hspace{7cm} \\ \cline{1-1}\cline{3-3} \\
	Ort, Datum && Unterschrift
	\end{tabular}

\end{document}
